\section{Dioden}
	Sperrstromdichte v. Si-Dioden: $10^{-11} \frac{A}{cm^2}$ \\
	\begin{tabular}{l l}
			\includegraphics[width=6cm]{./images/Diode-Kennlinie.png}
		&	\includegraphics[width=6cm]{./images/Diode-Kennlinie-2.png}
	\end{tabular}
	
	\subsection{Kleinsignal-ESB}
		\begin{tabular}{l l}
			\multirow{5}{*}{\includegraphics[width=4cm]{./images/Diode-KS-ESB.png}}
			& $R_B$: Bahnwiderstand (inkl. Zuleitung, Kontaktierung) \\
			& $r_d$: differentieller Widerstand des pn-Übergangs \\
			& $C_D$: Diffusionskapazität (bei pos. Diodenspannung) \\
			& $C_S$: Sperrschichtkapazität (bei neg. Diodenspannung)\\
			& $r_d=\frac{dU}{dI}$ (Tangentensteigung im AP) \\
		\end{tabular}
	
	\subsection{Grosssignal-ESB}
		\begin{tabular}{l l}
			\multirow{6}{*}{\includegraphics[width=4cm]{./images/Diode-GS-ESB.png}}
			& Gleichstromwiderstand $R_D = \frac{U_0}{I_0}$ (im AP)\\
			& $I=I_S \cdot (e^{\frac{U}{m \cdot U_T}}-1)$ \\
			& $I_S$: Sättigungssperrstrom (Bereich: $pA$)\\
			& $m$: Emissionskoeffizient (meistens: $m=1$) \\
			& $U_T=\frac{kT}{q} \approx 26mV$ \\
			& beim Umschalten bewirken die Kapazitäten $C_S$ und $C_D$ eine Verzögerung \\
		\end{tabular}
		
	\subsection{Temperaturverhalten}
		Faustregel: Durchflusspannung $U_{F0}$ ändert um $-2 \frac{mV}{K}$! Der Sperrstrom verdoppelt sich je Temperaturunterschied à $10K$ \\
		
		\textbf{Thermospannung $U_T$:} \\
		\begin{minipage}{6cm}
			\begin{align*}
				U_T &= \frac{k_B \cdot T}{q} \\
				U_{T_{23^{\circ}C}} &= 25,5 mV 
			\end{align*}
		\end{minipage}
		\begin{minipage}{10cm}
			\begin{align*}
			k_B &= 1,38065 \cdot 10^{-23} J/K &\text{Boltzmann-Konstante} \\
			q &= 1,60218 \cdot 10^{-19} As &\text{Elementarladung} \\
			\end{align*}
		\end{minipage}

	
\newpage	
	\subsection{Spezielle Dioden und Anwendungen}
		\subsubsection{Gleichrichter-Diode}
			\begin{minipage}[T]{6cm}
				\textbf{Einweg-Gleichrichter} \\
				\includegraphics[width=5cm]{./images/gleichrichter-einweg} \\
			\end{minipage}
			\begin{minipage}[T]{6cm}
				\textbf{Gleichrichter mit Glättung} \\
				\includegraphics[width=5cm]{./images/gleichrichter-einweg-glaettung} \\
				Bedingung: $\tau = R_L \cdot C_L \gg T$ \\
			\end{minipage}
			\begin{minipage}[T]{6cm}
				\textbf{Brücken-Gleichrichter} \\
				\includegraphics[width=6cm]{./images/gleichrichter-bruecke} \\
			\end{minipage}						

		\subsubsection{PIN-Diode}
			Für höhere Spannungsfestigkeit wird eigenleitende Halbleiterzone, die sog. 
			Intrinsic-Zone zwischen p- und n-dotierte Zone gelegt. Die Intrinsic-Zone
			ist im Sperrbetrieb sehr hochohmig und reduziert somit die el. Feldstärke
			auf ein akzeptables Mass.
		
		\subsubsection{Kapazitätsdiode}
			Kapazitätsdioden (Varicaps) werden in Sperrichtung betrieben. Die 
			Kapazität hängt dabei von der Spannung $U_{SP}$ ab.
			Anwendung: Einstellen der Schwingfrequenz in Oszillatoren \\
			
		\subsubsection{Tunneldiode}
			Tunneldioden besitzen eine sehr hohe Störstellenkonzentration und einen
			steilen Dotierungsverlauf. 
			
		\subsubsection{Zener-Diode}
			\begin{minipage}{6.5cm}
				\textbf{Kennline} \\
				\includegraphics[width=6cm]{./images/zdiode-kennlinie.png}
			\end{minipage}
			\begin{minipage}{4.5cm}
				\textbf{Grosssignal-ESB:} \\
				\includegraphics[width=4cm]{./images/zener-gs} \\
				\textbf{Kleinsignal-ESB:} \\
				\includegraphics[width=4cm]{./images/zener-ks}
			\end{minipage}
			\begin{minipage}{7.5cm}
				Anstieg der Durchbruchkennlinie bei spezieller Dotierung. Anwendung: Spannungsbegrenzung und -stabilisierung  \\
				
				Temperaturkoeffizient abhängig von Zenerspannung. $~0$ bei $5.6V$ -
				darüber: negativer Temperaturkoeffizient\\
			\end{minipage} \\
			
		\subsubsection{Schottky-Diode}
			Anstatt der p-Zone hat die Schottky-Diode eine Metall-Zone. Weil vom Metall
			keine Löcher in den Halbleiter diffundieren, gibt es keine Diffusionskapazität
			und die Diode bekommt ein sehr schnelles Schaltverhalten und eine kleine
			Durchflussspannung $U_{F0} \approx 0,3V$. \\
			
		\subsubsection{Leuchtdiode}
			Leuchtdioden werden aus verschiedenen Substraten (Halbleitern) hergestellt,
			z.B. GaAs (Gallium-Arsenid), InGaN (Indium-Gallium-Nitrid) um verschiedene
			Farben (Wellenlängen) zu produzieren. Typisch sind: \\
			\begin{tabular}{|l|l|l|c|} \hline
				infrarot & $900nm$ & GaAs & $1,0-1,5V$ \\ \hline
				rot & $635nm$ & GaAsP & $1,6-2.2V$ \\ \hline
				grün & $565nm$ & GaP & $2,0-2,4V$ \\ \hline
				blau & $490nm$ & InGaN & $3,2-4,0V$ \\ \hline
			\end{tabular}\\
			
		\subsubsection{Photodiode}
			Photodioden werden in Sperr-Richtung angeschlossen. Der Sperrstrom erhöht sich
			bei Lichteinfall. Meistens werden Photodioden mit einen OP-Verstärker (als
			Transimpedanzverstärker) benutzt. \\